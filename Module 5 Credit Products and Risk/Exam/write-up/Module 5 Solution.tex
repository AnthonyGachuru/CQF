\documentclass[a4paper,11pt] {article}
\usepackage{graphicx}
\usepackage{amssymb, amsmath, amsthm}
\usepackage{setspace}
\usepackage{amsfonts}

%-----------Margin, Linespread, Spacing-----------%
\usepackage[letterpaper, margin=1.3in]{geometry}
\usepackage[letterpaper]{geometry}
\linespread{1}
%-------------------------------------------------%

%-----------Define header and footnotes-----------
\usepackage{fancyhdr}               % Header and footnotes
\pagestyle{fancy}
\lhead{\bfseries \scriptsize CQF}
\chead{\bfseries \scriptsize Module 5 Solution}
\rhead{\bfseries \scriptsize Ran Zhao}
\renewcommand{\headrulewidth}{0.4pt}
%-------------------------------------------------%

%---------------------Listings--------------------%
\usepackage{listings}
\usepackage{color}
\definecolor{dkgreen}{rgb}{0,0.6,0}
\definecolor{gray}{rgb}{0.5,0.5,0.5}
\definecolor{mauve}{rgb}{0.58,0,0.82}
\lstset{ %
  language=Octave,                % the language of the code
  basicstyle=\footnotesize,           % the size of the fonts that are used for the code
  numbers=left,                   % where to put the line-numbers
  numberstyle=\tiny\color{gray},  % the style that is used for the line-numbers
  stepnumber=2,                   % the step between two line-numbers. If it's 1, each line
                                  % will be numbered
  numbersep=5pt,                  % how far the line-numbers are from the code
  backgroundcolor=\color{white},      % choose the background color. You must add \usepackage{color}
  showspaces=false,               % show spaces adding particular underscores
  showstringspaces=false,         % underline spaces within strings
  showtabs=false,                 % show tabs within strings adding particular underscores
  frame=single,                   % adds a frame around the code
  rulecolor=\color{black},        % if not set, the frame-color may be changed on line-breaks within not-black text (e.g. commens (green here))
  tabsize=2,                      % sets default tabsize to 2 spaces
  captionpos=b,                   % sets the caption-position to bottom
  breaklines=true,                % sets automatic line breaking
  breakatwhitespace=false,        % sets if automatic breaks should only happen at whitespace
  title=\lstname,                   % show the filename of files included with \lstinputlisting;
                                  % also try caption instead of title
  keywordstyle=\color{blue},          % keyword style
  commentstyle=\color{dkgreen},       % comment style
  stringstyle=\color{mauve},         % string literal style
  escapeinside={\%*}{*)},            % if you want to add LaTeX within your code
  morekeywords={*,...}               % if you want to add more keywords to the set
}
%-------------------------------------------------%

%----------------Title, Author, Dates-------------
\author{Ran Zhao}
\title{CQF Module 5 Exercise Solution}
\date{}
\begin{document}
\maketitle
%--------------------------------------------------


\textcolor{blue}{\bf 1 a} To compute the firm's asset value and volatility, set up the Merton type structural model as
\begin{eqnarray*}
E_0 &=& V_0  N(d_1) - D \exp (-rT) N(d_2) \\
d_1 &=& \frac{1}{\sigma_V} \left[ \log \left(\frac{V_0}{D} \right) + \left( r + \frac{1}{2}\sigma_V^2 \right) T \right] \\
d_2 &=& d_1 - \sigma_V \sqrt{T} \\
\sigma_E &=& \sigma_V N(d_1) \frac{V_0}{E_0}
\end{eqnarray*}

To solve the simultaneous equations numerically, I use MATLAB to find the minimum of the penalty function, where the deviations of $E_0$ and $\sigma_E$ between what are given in the context and computed results are calculated. The optimization results yield

$$ \left\{
\begin{aligned}
V_0 &=& 7.9088 \\
\sigma_V &=& 19.12\%
\end{aligned}
\right.
$$

Substitute the solutions into the simultaneous equations above, we yield back the equity value and equity volatility. The codes solving the equations are provided in the Appendix.

\bigskip

\textcolor{blue}{\bf 1 b} The probability of the default for Merton model is
$$
\mathbb{P}[V_t < D] = N(-d_2)
$$
whereas in the Black-Cox PD is calculated as
$$
\mathbb{P}[\tau \leq T | \tau > t] = N(h_1) + \exp\left\{ 2\left(r-\frac{\sigma_V^2}{2}\right) \log\left( \frac{K}{V_0} \right) \frac{1}{\sigma_V^2} \right\} N(h_2)
$$

Using the simultaneous equations in (1a) to solve  




\section*{Appendix}
\begin{spacing}{0.9}
\lstinputlisting[language=Matlab]{compute_E0.m}
\lstinputlisting[language=Matlab]{compute_value_vol_1a.m}
\end{spacing}

\end{document} 