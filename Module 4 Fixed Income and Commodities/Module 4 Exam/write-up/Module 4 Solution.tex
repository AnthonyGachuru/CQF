\documentclass[a4paper,11pt] {article}
\usepackage{graphicx}
\usepackage{amssymb, amsmath, amsthm}
\usepackage{setspace}
\usepackage{amsfonts}

%-----------Margin, Linespread, Spacing-----------%
\usepackage[letterpaper, margin=1.3in]{geometry}
\usepackage[letterpaper]{geometry}
\linespread{1}
%-------------------------------------------------%

%-----------Define header and footnotes-----------
\usepackage{fancyhdr}               % Header and footnotes
\pagestyle{fancy}
\lhead{\bfseries \scriptsize CQF}
\chead{\bfseries \scriptsize Module 4 Solution}
\rhead{\bfseries \scriptsize Ran Zhao}
\renewcommand{\headrulewidth}{0.4pt}
%-------------------------------------------------%

%---------------------Listings--------------------%
\usepackage{listings}
\usepackage{color}
\definecolor{dkgreen}{rgb}{0,0.6,0}
\definecolor{gray}{rgb}{0.5,0.5,0.5}
\definecolor{mauve}{rgb}{0.58,0,0.82}
\lstset{ %
  language=Octave,                % the language of the code
  basicstyle=\footnotesize,           % the size of the fonts that are used for the code
  numbers=left,                   % where to put the line-numbers
  numberstyle=\tiny\color{gray},  % the style that is used for the line-numbers
  stepnumber=2,                   % the step between two line-numbers. If it's 1, each line
                                  % will be numbered
  numbersep=5pt,                  % how far the line-numbers are from the code
  backgroundcolor=\color{white},      % choose the background color. You must add \usepackage{color}
  showspaces=false,               % show spaces adding particular underscores
  showstringspaces=false,         % underline spaces within strings
  showtabs=false,                 % show tabs within strings adding particular underscores
  frame=single,                   % adds a frame around the code
  rulecolor=\color{black},        % if not set, the frame-color may be changed on line-breaks within not-black text (e.g. commens (green here))
  tabsize=2,                      % sets default tabsize to 2 spaces
  captionpos=b,                   % sets the caption-position to bottom
  breaklines=true,                % sets automatic line breaking
  breakatwhitespace=false,        % sets if automatic breaks should only happen at whitespace
  title=\lstname,                   % show the filename of files included with \lstinputlisting;
                                  % also try caption instead of title
  keywordstyle=\color{blue},          % keyword style
  commentstyle=\color{dkgreen},       % comment style
  stringstyle=\color{mauve},         % string literal style
  escapeinside={\%*}{*)},            % if you want to add LaTeX within your code
  morekeywords={*,...}               % if you want to add more keywords to the set
}
%-------------------------------------------------%

%----------------Title, Author, Dates-------------
\author{Ran Zhao}
\title{CQF Module 4 Exercise Solution}
\date{}
\begin{document}
\maketitle
%--------------------------------------------------


\textcolor{blue}{\bf 1 } The zero coupon bonds satisfy
\begin{equation} \label{eqn::zero_bond}
\frac{\partial Z}{\partial t} + \frac{1}{2} \omega(r,t)^2 \frac{\partial^2 Z}{\partial r^2} + [u(r,t)-\lambda(r,t)\omega(r,t)] \frac{\partial Z}{\partial r} - rZ = 0
\end{equation}

As time being close to the maturity ($T-t\rightarrow 0$), expand the zero coupon bond with the form
\begin{equation} \label{eqn::zero_bond_expansion}
Z\sim 1 + a(r)(T-t) + b(r)(T-t)^2 + \ldots
\end{equation}

substitute $Z$ in Equation~\ref{eqn::zero_bond_expansion} into Equation~\ref{eqn::zero_bond}, and equating powers of $(T-t)$ yields
$$
Z(r,t;T)\sim 1 -r(T-t) + \frac{1}{2}(T-t)^2(r^2-u+\lambda\omega) + \ldots \qquad \textrm{as } t\rightarrow T
$$

So the shape of the yield curve near the short end becomes
$$
-\frac{\ln Z}{T-t} \sim r + \frac{1}{2}(u-\lambda\omega)(T-t) + \ldots \qquad \textrm{as } t\rightarrow T
$$

In Vasicek model, the risk-neutral spot rate take the form
$$
dr = (\eta-\gamma r) dt + \sqrt{\beta} dX
$$

and we have
$$
u-\lambda\omega = \eta-\gamma r
$$

Therefore, for the Vasicek model with one month Libor, we find
$$
r_L \sim r + \frac{1}{2}(\eta-\gamma r) \frac{1}{12}
$$

Finally, a floorlet cashflow has approximate value
$$
\max(r_f - r_L, 0) \sim \max\left(r_f - r - \frac{1}{24} (\eta-\gamma r), 0\right)
$$
as claimed.

\bigskip

\textcolor{blue}{\bf 2 } The Black-Derman \& Toy short-rate model is
\begin{equation} \label{eqn::BDT}
d(\log r) = \left( \theta(t) + \frac{d(\log\sigma(t))}{dt} \log r \right) dt + \sigma(t) dW
\end{equation}

Let $f(x) = \exp x$ and $X = \log r$. Apply It$\hat{o}$'s Lemma on $f(X)$ we obtain
$$
d(f(X)) = \frac{\partial f}{\partial X} dX + \frac{1}{2} \frac{\partial f^2}{\partial^2 X} dX^2
$$

where $\partial f / \partial X = \exp(X) = r$ and $\partial f^2 / \partial^2 X = \exp(X) = r$. Given $dX$ from Equation~\ref{eqn::BDT}, we then have
\begin{eqnarray*}
d(f(X)) &=& dr =  \frac{\partial f}{\partial X} dX + \frac{1}{2} \frac{\partial f^2}{\partial^2 X} dX^2 \\
        &=& r(t)\left( \theta(t) + \frac{d(\log\sigma(t))}{dt} \log r \right) dt + \sigma(t)r(t) dW + \frac{1}{2}\sigma^2(t) dt \\
        &=& r(t)\left( \theta(t) + \frac{d(\log\sigma(t))}{dt} \log r + \frac{1}{2} \sigma^2(t) \right) dt + \sigma(t)r(t) dW
\end{eqnarray*}

Using the notation of
$$
dr = A(r,t)dt + B(r,t)dW
$$

yields
$$
\left\{
  \begin{array}{ccl}
    A(r,t) & = & r(t)\left( \theta(t) + \frac{d(\log\sigma(t))}{dt} \log r + \frac{1}{2} \sigma^2(t) \right) \\
    B(r,t) & = & \sigma(t)r(t) \\
  \end{array}
\right.
$$

The terminal condition, $Z(r,T;T)=1$, must hold for all $r$, and this implies that $A(T,T)=B(T,T)=0$. 

\bigskip

\textcolor{blue}{\bf 3 } With model specification
$$
dr = [\eta(t)-\gamma r] dt + c dW
$$
where $\eta(t)$ is an arbitrary function of time $t$ and $\gamma$ and $c$ are constants. The partial differential
equation for the zero coupon bond price becomes
\begin{equation} \label{eqn::bond_hw}
\frac{\partial Z}{\partial t} + [\eta(t)-\gamma r]\frac{\partial Z}{\partial r} + \frac{1}{2}c^2 \frac{\partial^2 Z}{\partial r^2} = rZ
\end{equation}

Initially guess and subsequently verify that the solution has the form
$$
Z(r,t;T) = \exp(A(t;T)-rB(t;T))
$$
for some nonrandom functions $A(t;T)$ and $B(t;T)$ to be determined. Furthermore,
\begin{eqnarray*}
\frac{\partial Z}{\partial t}       &=& \left[\frac{\partial A(t,T)}{\partial t} - r\frac{\partial B(t,T)}{\partial t}\right]Z(r,t;T) \\
\frac{\partial Z}{\partial r}       &=& -B(t,T)Z(r,t;T) \\
\frac{\partial^2 Z}{\partial r^2}   &=& B(t,T)^2 Z(r,t;T)
\end{eqnarray*}

Substitution into the partial differential equation \ref{eqn::bond_hw} gives
\begin{equation} \label{eqn::eqn_bond_price}
\left[ \left( - \frac{\partial B(t,T)}{\partial t} + \gamma B(t,T) - 1 \right)r  \right. \\
+  \left. \frac{\partial A(t,T)}{\partial t}-\eta(t)B(t,T) +  \frac{1}{2}c^2 B(t,T)^2 \right]Z(r,t;T) = 0
\end{equation}

This equation must hold for all $r$, therefore the term that multiplies $r$ in this equation must be zero. Otherwise, changing the value of $r$ would change the value of the left-hand side of equation \ref{eqn::eqn_bond_price}, and hence it could not always be equal to zero. This gives us an ordinary differential equation in $t$ as
\begin{equation} \label{eqn::B_dyn}
\frac{\partial B(t,T)}{\partial t} = \gamma B(t,T) - 1
\end{equation}

Setting the term \ref{eqn::B_dyn} to zero in equation \ref{eqn::eqn_bond_price}, we have
\begin{equation} \label{eqn::A_dyn}
\frac{\partial A(t,T)}{\partial t} = \eta(t)B(t,T) -  \frac{1}{2}c^2 B(t,T)^2
\end{equation}

From the equation \ref{eqn::B_dyn}, it is not hard to solve
\begin{eqnarray*}
\frac{\partial B(t,T)}{\partial t} &=& \gamma B(t,T) - 1 \\
\frac{1}{\gamma} \frac{d (\gamma B(t,T) -1)}{\gamma B(t,T) -1} &=& d\tau \\
\frac{1}{\gamma} \int_t^T \frac{d (\gamma B(t,T) -1)}{\gamma B(t,T) -1} &=& \int_t^T d\tau \\
\frac{1}{\gamma} \log(\gamma B(t,T)-1) &=& - (T-t) \\
B(t,T) &=& \frac{1}{\gamma} \left( 1 - e^{-\gamma (T-t)} \right)
\end{eqnarray*}

Substitution the solution of $B(t,T)$ into \ref{eqn::A_dyn} yields
\begin{eqnarray*}
A(t,T) &=& - \left[ \int_t^T \eta(\tau)B(\tau,T) -  \frac{1}{2}c^2 B(\tau,T)^2 \right] d\tau \\
       &=& - \int_t^T \eta(\tau)B(\tau,T)d\tau + \frac{1}{2}c^2 \int_t^T \frac{1}{\gamma} \left( 1 - e^{-\gamma (T-t)} \right)^2 d\tau \\
       &=&  \int_t^T \eta(\tau)B(\tau,T)d\tau + \frac{c^2}{2\gamma^2} \left( (T-t) + \frac{2}{\gamma} e^{-\gamma(T-t)} -\frac{1}{2\gamma}e^{-2\gamma(T-t)} -\frac{3}{2\gamma} \right)
\end{eqnarray*}

And we solved $A(t,T)$ and $B(t,T)$ as claimed.

\bigskip

\textcolor{blue}{\bf 4 } Given the process
$$
dU_t = -\gamma U_t dt + \sigma dW_t
$$

we have
\begin{eqnarray*}
d(e^{\gamma }U_t) &=& \gamma e^{\gamma t} U_t dt + e^{\gamma t} d U_t \\
                  &=& \gamma e^{\gamma t} U_t dt - \gamma e^{\gamma t} U_t dt +  \sigma e^{\gamma t} dW_t \\
                  &=& \sigma e^{\gamma t} dW_t \\
e^{\gamma t} U_t  &=& U_0 + \sigma \int_{0}^{t}e^{\gamma s} dW_s \\
                  &=& u + \sigma \int_{0}^{t}e^{\gamma s} dW_s \\
U_t               &=& ue^{-\gamma t} + \sigma \int_0^t e^{-\gamma (t-s)} dW_s
\end{eqnarray*}

Then the moments of $U_t$ should be
\begin{eqnarray*}
\mathbb{E}[U_t] &=&  \mathbb{E}[ue^{-\gamma t}] + \mathbb{E}[\sigma \int_0^t e^{-\gamma (t-s)} dW_s] \\
                &=&  ue^{-\gamma t}  \\
\mathbb{V}[U_t] &=&  \mathbb{E}\left[ \sigma^2 \left( \int_0^t e^{-\gamma(t-s)}dW_s \right)^2 \right]  \\
                &=&  \sigma^2 \int_0^t e^{-2\gamma(t-s)} ds \\
                &=&  \frac{\sigma^2}{2\gamma} (1-e^{-2\gamma t})
\end{eqnarray*}

\bigskip

\textcolor{blue}{\bf 5 }



\end{document} 